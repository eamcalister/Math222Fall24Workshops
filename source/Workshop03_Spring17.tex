\documentclass[letterpaper,11pt]{examPTP}
%\documentclass[letterpaper,11pt,answers]{examPTP}
\usepackage{amsmath}
\usepackage{txfonts,graphicx}
\usepackage{pgf,tikz}
\usetikzlibrary{arrows}

\usepackage{headerPTP,diffeqPTP}

\renewcommand{\quizno}{Workshop 3}
\renewcommand{\quizdate}{February 28, 2017}
\renewcommand{\course}{Calc II (Math 222)}
\renewcommand{\instructorname}{}

\begin{document}
\pagestyle{headandfoot}


\noindent \underline{\hspace{6.5in}}


\begin{center}
\addpoints
\gradetable[h]  
\end{center}
\workshopinstructions

\begin{questions}
\question Suppose you want to purchase a house that costs $\$250,000$. The annual interest rate on your loan will be $3\%$ (so $0.25\%$ monthly), and the loan term will be 30 years and the monthly payment will be $M$ dollars. We want to solve for $M$. Assume you make your first payment at the end of the first month you have possession of the house.
\begin{parts}
\part[5] Write simplified formulas for the amount owed on the house, in terms of $M$, after you have owned the house for $1$, $2$, $3$, and $4$ months. You don't make a payment until the end of the first month.
\part[5] Use the pattern established in part (a) to write a simplified formula for the amount owed after you have had the house for $k$ months (where $k\leq 360$).
\part[5] The value of $M$ will be the value such that the total amount owed on the house after $360$ months is zero. Use this and the finite geometric series formula to solve for $M$.
\end{parts}  

\question  The goal of this problem is to write a repeating decimal as a fraction using a geometric series. We begin by recognizing a given geometric series as a repeating decimal. 
\begin{parts}
\part[5] Consider the geometric series $\displaystyle \sum_{n=0}^{\infty}789\left(\frac{1}{1000}\right)^n$. Write the first 4 partial sums of this series in decimal form (don't use a calculator). What repeating decimal does the infinite series represent?
\part[5] Use the formula for the sum of a geometric series to write the repeating decimal in part (a) as a fraction.
\part[5] Use ideas from part (a) and (b) to write $.\overline{9}$ as a fraction.
\end{parts} 



\end{questions}      

\vspace{0.5in}

\end{document}
